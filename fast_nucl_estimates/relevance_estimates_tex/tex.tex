% "Станет проще"

\documentclass[a4paper,12pt]{article} % тип документа

% report, book

% Рисунки
\usepackage{graphicx}
\usepackage{wrapfig}
\usepackage{lscape}
\usepackage{biblatex}
\usepackage{bbold}
\usepackage{physics}

\usepackage{hyperref}
\usepackage[rgb]{xcolor}
\hypersetup{				% Гиперссылки
    colorlinks=true,       	% false: ссылки в рамках
	urlcolor=blue          % на URL
}

%  Русский язык

%\usepackage[T2A]{fontenc}			% кодировка
\usepackage[utf8]{inputenc}			% кодировка исходного текста
%\usepackage[english,russian]{babel}	% локализация и переносы
\usepackage[english]{babel}


% Математика
\usepackage{amsmath,amsfonts,amssymb,amsthm,mathtools} 

\usepackage{wasysym}

\addbibresource{forCopy.bib}

%Заговолок
\author{Polyachenko Yury \\ CHM, Jacob's lab}
\title{CHM 5XX, task N}
\date{\today}

\begin{document} % начало документа

\newpage

I tried to reproduce what you did, but in 2D, and I think $N^*$ cancels.

My thoughts:

First, I think $\Gamma = \sqrt{-\eval{(F/T)''}_{N^*} / 2\pi}$, so the characteristic time $\tau$ is 

\begin{equation}
\tau \approx \frac{1}{D_N |\eval{(F/T)''}_{N^*}|} = \frac{1}{D_N 2 \pi \Gamma^2}
\end{equation}

Then, since 2D, we have $l \approx \rho^{-1/2}$, so

\begin{equation}
D_N \approx D / l^2 \approx D \rho
\end{equation}

Finally, $\pi R_d^2 \rho = N^*$, so

\begin{equation}
\tau_{nuc} \approx \frac{\ln(2 e^\gamma \beta \Delta F^*)}{D_N 2 \pi \Gamma^2} \ll \frac{R_d^2}{D}
\end{equation}

so

\begin{equation}
\frac{\ln(2 e^\gamma \beta \Delta F^*)}{D \rho 2 \pi \Gamma^2} \ll \frac{N^*}{\pi \rho D}
\end{equation}

so

\begin{equation}
\ln(2 e^\gamma \beta \Delta F^*) \ll 2 \Gamma^2 N^*
\end{equation}

Then, if we want to explicitly add CNT (we already assumed parabolic $\Delta F(N - N^*)$), then from 2D CNT

\begin{equation}
\Gamma \approx \sqrt{\frac{|\Delta \mu|}{4 \pi N^* T}} \Rightarrow 2 \Gamma^2 = \frac{|\Delta \mu|/T}{2 \pi N^*} = \frac{\ln S}{2 \pi N^*}
\end{equation}

so

\begin{equation}
\ln(2 e^\gamma \beta \Delta F^*) = \alpha \frac{\ln S}{2 \pi}, \hspace{10pt} \alpha = \frac{\tau_{nuc}}{\tau_{Diff}} \ll 1
\end{equation}

Then from 2D CNT 

\begin{equation}
\frac{\Delta F^*}{T} = \frac{\pi (\sigma/T)^2}{|\Delta \mu|/T} = \frac{\pi}{\ln S} (\sigma/T)^2
\end{equation}

which leads to

\begin{equation}
2 \pi e^\gamma \qty(\frac{\sigma}{T})^2 = S^{\alpha / 2 \pi} \ln S
\end{equation}

The conclusion for low surface tension $\sigma/T$ is the same. E.g. if $\alpha=0.1$ and $S=2$ then we need $\sigma/T < 0.25$ and if $S=1.3$ (I have someting like that for NVT-local according to my estimations) then $\sigma/T < 0.154$.

\end{document} % конец документа
